\PassOptionsToPackage{unicode=true}{hyperref} % options for packages loaded elsewhere
\PassOptionsToPackage{hyphens}{url}
%
\documentclass[]{article}
\usepackage{lmodern}
\usepackage{amssymb,amsmath}
\usepackage{ifxetex,ifluatex}
\usepackage{fixltx2e} % provides \textsubscript
\ifnum 0\ifxetex 1\fi\ifluatex 1\fi=0 % if pdftex
  \usepackage[T1]{fontenc}
  \usepackage[utf8]{inputenc}
  \usepackage{textcomp} % provides euro and other symbols
\else % if luatex or xelatex
  \usepackage{unicode-math}
  \defaultfontfeatures{Ligatures=TeX,Scale=MatchLowercase}
\fi
% use upquote if available, for straight quotes in verbatim environments
\IfFileExists{upquote.sty}{\usepackage{upquote}}{}
% use microtype if available
\IfFileExists{microtype.sty}{%
\usepackage[]{microtype}
\UseMicrotypeSet[protrusion]{basicmath} % disable protrusion for tt fonts
}{}
\IfFileExists{parskip.sty}{%
\usepackage{parskip}
}{% else
\setlength{\parindent}{0pt}
\setlength{\parskip}{6pt plus 2pt minus 1pt}
}
\usepackage{hyperref}
\hypersetup{
            pdfborder={0 0 0},
            breaklinks=true}
\urlstyle{same}  % don't use monospace font for urls
\usepackage{graphicx,grffile}
\makeatletter
\def\maxwidth{\ifdim\Gin@nat@width>\linewidth\linewidth\else\Gin@nat@width\fi}
\def\maxheight{\ifdim\Gin@nat@height>\textheight\textheight\else\Gin@nat@height\fi}
\makeatother
% Scale images if necessary, so that they will not overflow the page
% margins by default, and it is still possible to overwrite the defaults
% using explicit options in \includegraphics[width, height, ...]{}
\setkeys{Gin}{width=\maxwidth,height=\maxheight,keepaspectratio}
\setlength{\emergencystretch}{3em}  % prevent overfull lines
\providecommand{\tightlist}{%
  \setlength{\itemsep}{0pt}\setlength{\parskip}{0pt}}
\setcounter{secnumdepth}{0}
% Redefines (sub)paragraphs to behave more like sections
\ifx\paragraph\undefined\else
\let\oldparagraph\paragraph
\renewcommand{\paragraph}[1]{\oldparagraph{#1}\mbox{}}
\fi
\ifx\subparagraph\undefined\else
\let\oldsubparagraph\subparagraph
\renewcommand{\subparagraph}[1]{\oldsubparagraph{#1}\mbox{}}
\fi

% set default figure placement to htbp
\makeatletter
\def\fps@figure{htbp}
\makeatother

\makeatletter
\@ifpackageloaded{subfig}{}{\usepackage{subfig}}
\@ifpackageloaded{caption}{}{\usepackage{caption}}
\captionsetup[subfloat]{margin=0.5em}
\AtBeginDocument{%
\renewcommand*\figurename{Figure}
\renewcommand*\tablename{Table}
}
\AtBeginDocument{%
\renewcommand*\listfigurename{List of Figures}
\renewcommand*\listtablename{List of Tables}
}
\@ifpackageloaded{float}{}{\usepackage{float}}
\floatstyle{ruled}
\@ifundefined{c@chapter}{\newfloat{codelisting}{h}{lop}}{\newfloat{codelisting}{h}{lop}[chapter]}
\floatname{codelisting}{Listing}
\newcommand*\listoflistings{\listof{codelisting}{List of Listings}}
\@ifpackageloaded{cleveref}{}{\usepackage{cleveref}}
\crefname{figure}{fig.}{figs.}
\Crefname{figure}{Fig.}{Figs.}
\crefname{table}{tbl.}{tbls.}
\Crefname{table}{Tbl.}{Tbls.}
\crefname{equation}{eq.}{eqns.}
\Crefname{equation}{Eq.}{Eqns.}
\crefname{listing}{lst.}{lsts.}
\Crefname{listing}{Lst.}{Lsts.}
\crefname{section}{sec.}{secs.}
\Crefname{section}{Sec.}{Secs.}
\crefname{codelisting}{\cref@listing@name}{\cref@listing@name@plural}
\Crefname{codelisting}{\Cref@listing@name}{\Cref@listing@name@plural}
\makeatother

\date{}

\begin{document}

\hypertarget{example-of-two-subfig-figures}{%
\section{Example of two subfig
figures}\label{example-of-two-subfig-figures}}

\begin{figure}
\centering

\subfloat[The yellow guy with square
pants.]{\includegraphics[width=0.48\textwidth,height=\textheight]{spongebob.png}\label{fig:spongebob}}
\hfill \subfloat[The dim but loveable pink sea
star.]{\includegraphics[width=0.48\textwidth,height=\textheight]{patrick.png}\label{fig:patrick}}

\caption{Two magnificent creatures of the sea.}

\label{fig:sbsp}

\end{figure}

Best buddies in \cref{fig:sbsp}. The protagonist in \cref{fig:spongebob}
and his trusty sidekick in \cref{fig:patrick}.

\cleardoublepage

\hypertarget{example-of-three-subfig-figures}{%
\section{Example of three subfig
figures}\label{example-of-three-subfig-figures}}

\begin{figure}
\centering

\subfloat[Rainbow
Dash.]{\includegraphics[width=0.3\textwidth,height=\textheight]{img1.png}\label{fig:rainbow-dash}}
\hfill \subfloat[Twilight
Sparkle.]{\includegraphics[width=0.3\textwidth,height=\textheight]{img2.png}\label{fig:twilight-sparkle}}
\hfill \subfloat[Pinkie
Pie.]{\includegraphics[width=0.3\textwidth,height=\textheight]{img3.png}\label{fig:pinkie-pie}}

\caption{Look at my pretty horses!}

\label{fig:horses}

\end{figure}

We can refer to these fine stallions in several ways:

\begin{itemize}
\tightlist
\item
  \textbf{Refer to subfig labels:} Notice the awesome stripes of
  \cref{fig:twilight-sparkle}, and the bushy tail of
  \cref{fig:pinkie-pie} and finally the colorful tail of
  \cref{fig:rainbow-dash}.
\item
  \textbf{Refer to range of labels:} Look at all the pretty horses in
  subfigures
  \cref{fig:twilight-sparkle,fig:pinkie-pie,fig:rainbow-dash}.
\item
  \textbf{Refer to overall figure:} Or simply behold the sight of
  \cref{fig:horses}.
\end{itemize}

\end{document}
